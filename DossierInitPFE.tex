% Options possibles
% init - pour le dossier d'initialisation
% francais ou english
% RandD - réservé aux PFE ayant le label R&D.
% Confidential - réservé aux PFE confidentiels
\documentclass[init,francais]{rapportPFE}  % pour une version française
%\documentclass[english]{rapportPFE}  % pour une version avec du texte en anglais
% L'option confidentiel a pour conséquence d'ajouter le filigrane "Confidentiel" sur la première et la dernière page
%  En fait, ce filigrane apparait sur toutes les pages numérotées 1 par Latex… 
% Pour le rajouter sur TOUTES les pages, décommenter la commande suivante
%\newwatermark[allpages,textalign=center,fontfamily=pbk,angle=55,scale=2.75,xpos=-1cm,ypos=1cm,color=lightgray]{Confidentiel}

%
\usepackage{listings}

% %%%%%%%%%%%%%%%%%%%%%%%%%%%%%%%%%%%%%%%%%%%
% titres du rapport
% %%%%%%%%%%%%%%%%%%%%%%%%%%%%%%%%%%%%%%%%%%%
% Version francaise
\titre{Sécurisation d'un logiciel de validation / génération de données}
% version anglaise
\title{Securing a software for validation/data generation.}


% %%%%%%%%%%%%%%%%%%%%%%%%%%%%%%%%%%%%%%%%%%%
% Auteur du rapport
% %%%%%%%%%%%%%%%%%%%%%%%%%%%%%%%%%%%%%%%%%%%
\firstname{Adrien}
%\middlename{Autre-Prénom}
\lastname{Jaillet}



% %%%%%%%%%%%%%%%%%%%%%%%%%%%%%%%%%%%%%%%%%%%
% Informations sur le PFE
% %%%%%%%%%%%%%%%%%%%%%%%%%%%%%%%%%%%%%%%%%%%
% Information administratives
\dateDebutPFE{6 février 2023}
\dateFinPFE{28 juillet 2023}
\nomStructureAcceuil{Clearsy}
\villeStructureAccuel{Aix-en-provence}
% \logoStructureAccueil{width=1.5cm}{graphics/LogoStructureAccueil}% if you don't want to put a logo for the host structure, comment this line.

% Encadrants.
% Penser à accorder la fonction avec le genre
\begin{encadrants}
% Fonction {Fonction}{ Prénom \Nom{Nom}}{Titre}{Structure}
  \referent{Référente}{Sahra \Nom{Bouchenak}}{Professeure}{INSA Lyon}%
  \tuteur{Tuteur}{Romain \Nom{Lapostolle}}{Ingénieur en Informatique}{Clearsy}
\end{encadrants}

% Date de la soutenance
\date{juin 2020}



% %%%%%%%%%%%%%%%%%%%%%%%%%%%%%%%%%%%%%%%%%%%
% Rapport lui même
% %%%%%%%%%%%%%%%%%%%%%%%%%%%%%%%%%%%%%%%%%%%
\begin{document}
\maketitle


% ==========================================
% les résumés et les mots clés du rapport
% ==========================================

\begin{ResumeMotsCles}

% Version anglaise
% abstract no more than 12 lines.
\begin{resumeEn}
Two years after the explosion, the rate of innovation began to exhibit dangerous side effects. The explosive growth had provided an exciting validation of the collaborative hacker approach, but it had also led to over-complexity. "We had a Tower of Babel effect," says Guy Steele.\\
Such statements, while reflective of the hacker ethic, also reflected the difficulty of translating the loose, informal nature of that ethic into the rigid, legal language of copyright. In writing the GNU Emacs License, Stallman had done more than close up the escape hatch that permitted proprietary offshoots. He had expressed the hacker ethic in a manner understandable to both lawyer and hacker alike.\\
The German sociologist Max Weber once proposed that all great religions are built upon the "routinization" or "institutionalization" of charisma. Every successful religion, Weber argued, converts the charisma or message of the original religious leader into a social, political, and ethical apparatus more easily translatable across cultures and time.

\end{resumeEn}
% keywords
\keywords{Eruditos~; obscuros~; occulte~; provinciae~; atrocium.}



% Version francaise
% Résumé pas plus de 12 lignes
\begin{resumeFr}
Celles-ci sont par extraordinaire à huit heures. Au-delà des autorités, tout le paquet de linge blanc. Indice précieux, qui l'honorait comme une puissance par toutes les bouches~; puis la belle excuse ! Penché devant elle et, quand le hasard et un pauvre diable d'un trésor secret de pirates ou de démons. Saisissant son fusil, et puis la catastrophe qui l'avait prise sur le mur. Amour, tu ne me verras plus. Fumer la cigarette, elle redoublait ses étouffements. Soucieux de conserver toute sa vie à mal faire. 
Courir sur la pointe du couteau de la rue en son costume~; j'en fus témoin. Ainsi il n'était même pas sérieusement importuné par son banquier. Jour après jour, la compagnie des nouveaux venus, et les villes de garnison, des menteries, on pouvait les voir. Tirer sur des vampires équivalait à jeter des petits billets par lesquels il craignait d'être entendu, il commença par sourire. Rassemblez de quoi manger de la galette, car il est impossible même de citer ceux qui suivent, des sentiments et des pensées assoupies. Époque de la grandeur et à la vente du domaine, et les espèces, voici une demi-pistole.
\end{resumeFr}

% Mots clés
\motscles{Eruditos~; obscuros~; occulte~; provinciae~; atrocium.}
\end{ResumeMotsCles}


% ==========================================
% Remerciements éventuels
% ==========================================
% \begin{remerciements}
%   Merci à tous. Commenter cet environnement s'il n'est pas nécessaire.
% \end{remerciements}





% %%%%%%%%%%%%%%%%%%%%%%%%%%%%%%%%%%%%%%%%%%
% rapport proprement dit 
% %%%%%%%%%%%%%%%%%%%%%%%%%%%%%%%%%%%%%%%%%%

% ==========================================
% Sommaire (généré automatiquement)
% ==========================================
% \setcounter{tocdepth}{3}
\tableofcontents
% \cleardoublepage

% ==========================================
% Introduction
% ==========================================
%Contexte, 
%définition du problème, 
%aperçu des contributions, 
%plan du rapport

\section{Introduction}

% ==========================================
% Présentation
% ==========================================
\section{Présentation du Projet de Fin d'Études.}
\subsection{Contexte.}
ClearSy est une entreprise française spécialisée dans la conception, le développement et l'assurance de la sécurité des systèmes critiques 
pour des industries telles que le ferroviaire, l'aéronautique et la défense. 
Ils fournissent des services tels que l'ingénierie système, l'ingénierie logicielle, la conception matérielle, la vérification et la validation, ainsi que l'analyse de sécurité. ClearSy a également développé plusieurs outils basés sur des méthodes formelles, qui sont utilisés dans le développement de systèmes critiques.
\\~\\
ClearSy a développé un outil de validation de données critiques nommé CAVAL, qui utilise des méthodes formelles. L'objectif de l'entreprise est d'améliorer ce logiciel en proposant un module sécurisé de génération de données basé sur les contraintes fournies par l'utilisateur. Bien que ce module soit partiellement implémenté, il n'est pas encore assez sécurisé pour une utilisation industrielle dans un contexte ferroviaire, qui requiert le plus haut niveau de sécurité fourni par la norme européenne CENELEC EN 50128.


\subsection{Objectifs.}

Le stagiaire sera chargé de travailler sur un nouveau module. L'analyse de sécurité, identifiant tous les potentiels risques ainsi que les parties du logiciel à sécuriser pour respecter les normes en vigueur, est en partie déjà réalisée.
Le stagiaire sera responsable de mettre en place les couches de sécurité nécessaires pour respecter les normes imposées. Il devra choisir les méthodes de sécurisations les plus appropriées pour chaque partie du logiciel, en veillant à ne pas affecter les performances de ce dernier.
\\~\\
Le stagiaire travaillera plus spécifiquement sur les modules d'import de données. Étant donné que les clients peuvent stocker leurs données dans différents formats (CSV, XML, EXCEL, etc.),
le logiciel doit être capable d'importer des données provenant de ces différents formats tout en maintenant l'intégrité des données. 
Les méthodes formelles n'étant pas adaptées à la sécurisation de l'import de données, le stagiaire devra redonder certaines partie du logiciel.

\subsubsection{Tâches à réaliser.}

\paragraph*{Récupération de données dans les fichiers Excel}


Le stagiaire aura pour mission de mettre en place le module d'import de données pour les fichiers Excel, 
qui doit être capable d'importer des données de différents formats tels que xlsx, xls, xlsm ou csv. 
Pour ce faire, il devra définir une description de schéma pour les fichiers Excel (similaire à un fichier XSD pour les fichiers XML). 
Une fois cette description établie, il devra mettre en œuvre l'algorithme permettant de structurer les données des fichiers dans des structures de données C++ 
afin que CAVAL puisse les traiter.
Le travail effectué sera validé à l'aide d'un ensemble de tests déjà existants.\\~\\
Livrables:
\begin{itemize}
  \item Code source
  \item Documentation
\end{itemize}\\~\\
Date Limite:
\begin{itemize}
  \item 25 mars
\end{itemize}

\paragraph*{Création d'un module d'import externe à CAVAL pour appeler des programmes sachant importer des données}


Le stagiaire aura pour tâche de créer un module d'importation externe à CAVAL qui permettra d'appeler des programmes
capables d'importer des données. Ce module devra être capable de communiquer avec ces programmes via une interface définie par le stagiaire afin d'externaliser l'importation des données hors de CAVAL.
\\~\\
Livrables:
\begin{itemize}
  \item Code source
  \item Documentation
\end{itemize}\\~\\
Date Limite:
\begin{itemize}
  \item 8 avril
\end{itemize}

\paragraph*{Redondance du module d'import de données pour les fichiers textes en C#}


Le stagiaire aura pour mission de ré-implementer le module d'importation de données pour les fichiers texte en C#, en utilisant la description de schéma des fichiers texte déjà existante.
\\~\\
Livrables:
\begin{itemize}
  \item Code source
  \item Documentation
\end{itemize}\\~\\
Date Limite:
\begin{itemize}
  \item 1 mai
\end{itemize}


\paragraph*{Autre tâches}

Un fois que ces tâches seront réalisées, le stagiaire choisira ce qu'il veut faire parmis les tâches suivantes:
\begin{itemize}
  \item Compiler CAVAL avec un autre compilateur
  \item Redonder le module d'import des données à partir de fichiers xml en utilisant le format xsd comme description de schéma
  \item Redonder le module d'import JSON
\end{itemize}



\subsection{Environnement scientifique et technique.}
Le stagiaire sera immergé dans un environnement de travail spécifique à l'entreprise, 
qui met en place de nombreux processus pour garantir la qualité de ses logiciels. 
Il apprendra à travailler en respectant les normes et les standards imposés. Son travail sera soumis validation à et fera l'objet d'une certification.



% Bibliographie
% Annexes éventuelles (en plus des 30 pages demandées)
% \bibliographystyle{unsrt}
% \bibliography{rapportPFE}


\end{document}
